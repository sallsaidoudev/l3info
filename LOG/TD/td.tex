\documentclass[a4paper]{article}

\usepackage{a4wide}
\usepackage[utf8]{inputenc}
\usepackage[T1]{fontenc}
\usepackage[francais]{babel}

\usepackage{amssymb}
\usepackage{graphicx}
\usepackage[usenames,dvipsnames]{color}

\usepackage{hyperref} \urlstyle{sf}
\hypersetup{
  colorlinks=true,
  urlcolor=BlueViolet,
  citecolor=BlueViolet,
  linkcolor=BlueViolet,
}
\DeclareUrlCommand\email{\urlstyle{sf}}

\setlength{\parindent}{0pt}
\usepackage{parskip}

\renewcommand{\thefootnote}{\arabic{footnote}}
\newcounter{nbdefi}
\newenvironment{defi}
  {\stepcounter{nbdefi} \par \textbf{def \thenbdefi}~~~}{}
\newcounter{nblem}
\newenvironment{lem}
  {\stepcounter{nblem} \par \textbf{lem \thenblem}~~~}{}
\newcounter{nbprop}
\newenvironment{prop}
  {\stepcounter{nbprop} \par \textbf{prop \thenbprop}~~~}{}
\newenvironment{ex}{\textit{Exemple}\begin{quote}}{\end{quote}}
\newcommand{\rem}[1]{~\\ \textit{Remarque ---} #1}
\newcommand{\reml}[1]{\textit{Remarque ---} #1}

\usepackage{listings}
\lstset{
  language=C,
  basicstyle=\ttfamily,
  keywordstyle=\color{OliveGreen},
  stringstyle=\color{Bittersweet},
  showstringspaces=false,
  commentstyle=\color{Gray},
  numbers=left,
  numberstyle=\ttfamily\color{Gray},
  frame=l,
  columns=fullflexible,
  rulecolor=\color{Gray},
  tabsize=4,
  extendedchars=true,
  literate=
	{É}{{\'E}}1 {è}{{\`e}}1 {à}{{\`a}}1 {È}{{\`E}}1 {À}{{\`A}}1 {ê}{{\^e}}1 {â}{{\^a}}1 {î}{{\^\i}}1 {ô}{{\^o}}1
	{Ê}{{\^E}}1 {Â}{{\^A}}1 {Î}{{\^I}}1 {Ô}{{\^O}}1 {Û}{{\^U}}1 {ë}{{\"e}}1 {ï}{{\"\i}}1 {ü}{{\"u}}1 {Ë}{{\"E}}1
	{Ï}{{\"I}}1 {Ü}{{\"U}}1 {û}{{\^u}}1 {ç}{{\c c}}1 {Ç}{{\c C}}1 {æ}{{\ae}}1 {Æ}{{\AE}}1 {œ}{{\oe}}1 {Œ}{{\OE}}1
	{é}{{\'e}}1,
}


\newcounter{noexo}
\newcommand{\exo}{\stepcounter{noexo} \paragraph{Exercice \thenoexo} ~\\ ~\\}

\title{Logique -- TD}
\author{Sophie Pinchinat -- L3 Info Rennes 1}
\date{2015 -- 2016, S2}

\begin{document}

\maketitle

\subsection*{TD1}

\exo
Soit $\phi$ valide, $mod(\phi) = Val$ et comme (cours) $mod(\neg\phi)=Val\setminus mod(\phi)= Val\setminus Val=\emptyset$, $\neg\phi$ n'est pas satisfiable. Réciproquement, $mod(\neg\phi) = \emptyset \Rightarrow mod(\phi) = mod(\neg\neg\phi) = Val\setminus mod(\neg\phi) = Val\setminus\emptyset = Val$ et $\phi$ est valide ; d'où (1).

$\exists\nu\in mod(\phi) \Leftrightarrow mod(\neg\phi) = Val\setminus mod(\phi) \subsetneq Val$ (2)

\exo
$var(\phi)\subseteq X$ et $\nu_X=\nu'_X$ donc $\nu_{var(\phi)}=\nu'_{var(\phi)}$, d'où $\nu(\phi) = \nu'(\phi)$.

\exo
Pour toute proposition $\phi\equiv p$, $\phi^*\equiv\neg p\equiv\neg\phi$ donc la propriété est vraie. Soient $\phi$ et $\psi$ qui vérifient la propriété, alors $\phi\vee\psi$ se transforme en $\neg\phi\wedge\neg\psi$ et $\neg(\neg\phi\wedge\neg\psi)\equiv\neg\neg(\phi\vee\psi)\equiv\phi\vee\phi$ (calcul identique pour $\wedge$ transformé en $\vee$). Donc la propriété est vraie pour toute formule.

\exo
Tout d'abord, on a $p\rightarrow q\equiv (\neg p)\vee q$ donc toute formule contenant le signe $\rightarrow$ peut s'écrire sans, et $\{\neg,\wedge,\vee\}\equiv\{\neg,\wedge,\vee,\rightarrow\}$. Ensuite, comme vu précédemment, $p\wedge q=\neg(\neg p \vee \neg q)$ donc de même, $\{\neg, \vee\}\equiv\{\neg,\wedge,\vee\}$ (+transitivité). Ajoutons que $\{\neg\}\sqsubset\{\neg,\vee\}$ (évident) et on obtient $$\{\neg\}\sqsubset\{\neg,\vee\}\equiv\{\neg,\wedge,\vee,\rightarrow\}$$ (et plus généralement, $\neg$ et $\vee$ (ou $\wedge$, calculs analogues) suffisent à écrire toute formule de calcul propositionnel).

\subsection*{TD2}
\setcounter{noexo}{0}

\exo
$\Gamma\models\varphi$ donne par définition $mod(\Gamma)\subseteq mod(\varphi)$. Soit $\nu$ une valuation modèle de $\Gamma\cup\{\neg\varphi\}$ ; cette valuation est aussi un modèle du sous-ensemble $\Gamma$ et donc de $\varphi$ d'après l'inclusion précédente. Alors $\nu(\varphi)=1$ et $\nu(\neg\varphi)=1$, ce qui est impossible. Il n'existe donc aucun modèle pour $\Gamma\cup\{\neg\varphi\}$, cet ensemble de formules est contradictoire.

Réciproquement, soit $\nu$ un modèle de $\Gamma$ ; si $\nu(\varphi)=0$, alors $\nu(\neg\varphi)=1$ et $\nu$ satisfait $\Gamma\cup\{\neg\varphi\}$, ce qui est impossible. Donc $\nu(\varphi)=1$, $\nu$ est un modèle de $\varphi$ et on a bien $\Gamma\models\varphi$.

\exo
Soit $\nu\in mod(\Sigma\cup\Gamma)$ ; $\forall\psi\in\Sigma, \psi\in\Sigma\cup\Gamma$ donc $\nu(\psi)=1$ donc $\nu\in mod(\Sigma)$ et par un calcul similaire, $\nu\in mod(\Gamma)$. Ainsi, $\nu\in mod(\Sigma)\cap mod(\Gamma)$.

Réciproquement, soit $\nu\in mod(\Sigma)\cap mod(\Gamma)$. $\forall\varphi\in\Sigma\cup\Gamma$, si $\varphi\in\Sigma$, alors $\nu(\varphi)=1$ puisque $\nu\in mod(\Sigma)$ ; si $\varphi\in\Gamma$, le calcul est analogue et donc on a bien $\nu\in mod(\Sigma\cup\Gamma)$. Ainsi, $mod(\Sigma\cup\Gamma)= mod(\Sigma)\cap mod(\Gamma)$.

\exo
1. $\Gamma_2=\{\varphi_1,\varphi_2\}$ est une telle simplification de $\Gamma_1$ ; en effet, tout modèle de $\Gamma_1$ est trivialement un modèle de $\Gamma_2$ et réciproquement, tout modèle de $\Gamma_2$ force $p_B$ vrai et $p_C$ faux, ce qui valide $\varphi_3$, et est donc aussi un modèle de $\Gamma_1$.

2. $mod(\Gamma_2)=\{(p_A, p_B, \neg p_C, p_D), (p_A, p_B, \neg p_C, \neg p_D), (\neg p_A, p_B, \neg p_C, p_D)\}$

3. $\Gamma_2$ est consistant (il admet trois modèles).

4. $\varphi_3$ est une conséquence logique de $\Gamma_2$.

5. Il est contradictoire (aucun modèle de $\Gamma_2$ ne valide la nouvelle proposition, du coup ça marche pas).

\exo
$T$ : je joue au tennis, $R$ : je regarde du tennis à la télé, $A$ : je lis des articles sur le tennis. La publicité s'énonce $(\neg T\rightarrow R) \wedge (\neg R \rightarrow A) \equiv (T \vee R)\wedge(R\vee A) \equiv R\vee(T\wedge A)$ ; ainsi, la seule activité possible à l'exclusion des deux autres est $R$.

\exo
$A, B, C$ pour noter chaque suspect ; $D$ : Brown connaissait la victime ; $E$ : Clark détestait la victime ; $F$ : Brown était en ville ; $G$ : Adams était en ville. Les déclarations des suspects se formulent comme suit :\begin{itemize}
  \item $\varphi_A = \neg A \wedge D \wedge E$
  \item $\varphi_B = \neg B \wedge \neg D \wedge \neg F$
  \item $\varphi_C = \neg C \wedge F \wedge G \wedge (A \vee B)$
\end{itemize}
Si Adams est le coupable, alors Brown et Clark disent la vérité d'après l'énoncé ; or, leurs déclarations sont contradictoires (l'un affirme $F$ et l'autre $\neg F$). Si Clark est le coupable, le même problème se pose entre les déclarations d'Adams et Brown. C'est donc Brown le coupable, ce saligaud.

\exo
1. En numérotant les position de 1 à 4 dans le sens horaire en partant du haut, on peut exprimer chaque contrainte sous la forme : $1A$ signifie "la carte à la position 1 est un As", ou $3\blacklozenge$ pour "la carte à la position 3 est un Carreau".

2. Notre problème se modélise alors par un ensemble de clauses d'unicité : $(1A\wedge\neg 2A\wedge\neg 3A\wedge\neg 4A)\vee(\neg 1A\wedge 2A\wedge\neg 3A\wedge\neg 4A) \vee...$ (avec le même genre de clauses pour les couleurs), auxquelles on adjoint les contraintes de l'énoncé :\begin{itemize}
\item $1A$,
\item $3\spadesuit$,
\item $2D$,
\item $1R\rightarrow 1\clubsuit \wedge 2R\rightarrow 2\clubsuit \wedge ...$, 
\item $1A\rightarrow \neg 1\blacklozenge \wedge 2A\rightarrow \neg 2\blacklozenge \wedge...$
\end{itemize}
\end{document}
