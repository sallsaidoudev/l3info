\documentclass[a4paper,11pt]{article}

\usepackage[utf8]{inputenc}
\title{ARC2 -- TP 8}
\author{Prune Forget, Léo Noël-Baron \& Thierry Sampaio}
\date{23/03/2016}

\usepackage{a4wide}
\usepackage{textcomp}
\usepackage[utf8]{inputenc}
\usepackage[T1]{fontenc}
\usepackage[francais]{babel}

\usepackage{graphicx}
\usepackage[usenames,dvipsnames]{color}

\usepackage{hyperref} \urlstyle{sf}
\hypersetup{
  colorlinks=true,
  urlcolor=BlueViolet,
  citecolor=BlueViolet,
  linkcolor=BlueViolet,
}
\DeclareUrlCommand\email{\urlstyle{sf}}

\newenvironment{keywords}
  {\description\item[\bsc{Mots-clés}]~$\cdot$~ }
  {\enddescription}
\newenvironment{remarque}
  {\description\item[\bsc{Remarque} ---]\sl}
  {\enddescription}
\renewcommand{\thefootnote}{\arabic{footnote}}

\usepackage{listings}
\lstset{
  language=C,
  basicstyle=\ttfamily,
  keywordstyle=\color{OliveGreen},
  stringstyle=\color{Bittersweet},
  showstringspaces=false,
  commentstyle=\color{Gray},
  numbers=left,
  numberstyle=\ttfamily\color{Gray},
  frame=l,
  columns=fullflexible,
  rulecolor=\color{Gray},
  tabsize=4,
  extendedchars=true,
  literate=
	{É}{{\'E}}1 {è}{{\`e}}1 {à}{{\`a}}1 {È}{{\`E}}1 {À}{{\`A}}1 {ê}{{\^e}}1 {â}{{\^a}}1 {î}{{\^\i}}1 {ô}{{\^o}}1
	{Ê}{{\^E}}1 {Â}{{\^A}}1 {Î}{{\^I}}1 {Ô}{{\^O}}1 {Û}{{\^U}}1 {ë}{{\"e}}1 {ï}{{\"\i}}1 {ü}{{\"u}}1 {Ë}{{\"E}}1
	{Ï}{{\"I}}1 {Ü}{{\"U}}1 {û}{{\^u}}1 {ç}{{\c c}}1 {Ç}{{\c C}}1 {æ}{{\ae}}1 {Æ}{{\AE}}1 {œ}{{\oe}}1 {Œ}{{\OE}}1
	{é}{{\'e}}1,
}
%\lstMakeShortInline{|}

\parskip=0.3\baselineskip
\sloppy

\makeatletter
  \let\runtitle\@title
  \let\runauthor\@author
\makeatother

\usepackage{fancyhdr}
\pagestyle{fancy}
\fancyhead{}
\lhead{\runtitle}
\rhead{\runauthor}
\setlength{\headheight}{13.6pt}


\begin{document}

\maketitle

Ce TP propose d'utiliser les mécanismes d'interruption du Nios pour gérer des événements matériels (ici une souris et une animation à l'écran).

\subsection*{Activation des interruptions}

On commence par implémenter les fonctions d'activation :
\begin{lstlisting}[language=C]
void activer_interruptions() {
	unsigned int tmp;
	NIOS2_READ_STATUS(tmp);
	tmp |= 1; // Mettre le bit 0 de STATUS à 1
	NIOS2_WRITE_STATUS(tmp);
}\end{lstlisting}
\begin{lstlisting}[language=C]
void activer_interruption(int num) {
	unsigned int tmp;
	NIOS2_READ_IENABLE(tmp);
	tmp |= (1 << num); // Mettre le bit num de IENABLE à 1
	NIOS2_WRITE_IENABLE(tmp);
}\end{lstlisting}
et on appelle la première au début du \verb?main?.

\subsection*{Contrôle de l'animation}

Pour gérer l'animation à l'écran, on doit lancer le timer en mode continu soit, d'après l'énoncé, \verb?0111? ce qui donne \verb?*TIMER_CTRL = 7? dans le \verb?main?. On doit ensuite appeler \verb?activer_interruption(INTERVAL_TIMER_IRQ)? pour activer l'interruption du timer. Enfin, il faut implémenter la routine d'interruption :
\begin{lstlisting}[language=C]
static void TIMER_ISR(void *context, alt_u32 id) {
	tick();
	*TIMER_STATUS = 0;
}\end{lstlisting}
et ô miracle, les nuages défilent.

\subsection*{Contrôle de la souris}

Pour gérer la souris, il faut d'abord activer l'interruption correspondante dans le \verb?main? puis écrire la routine suivante :
\begin{lstlisting}[language=C]
char pnext = 0, p[3];
static void MOUSE_ISR(void *context, alt_u32 id)  {
	p[pnext] = (*PS2_DATA) & 0xFF; // Récupérer un octet de données
	if(++pnext == 3) { // Si on a eu les deux déplacements
		x_pos = MOD(x_pos + p[1], 320);
		y_pos = MOD(y_pos - p[2], 240);
		pnext = 0; // RAZ
	}
}\end{lstlisting}
qui lit les octets de données de la souris un à un et met à jour la position du curseur quand il le faut. Et ô joie, le canari vole.

\newpage
\subsection*{main.c}
\begin{lstlisting}[language=C]
#include "nios2.h"
#include "system.h"
#include "sys/alt_irq.h"
#include "init.h"

#include "graphlib.h"
#include "cursor.h"
#include "nuages.h"

#define MOD(a,b) ((a)%(b)+(b))%(b)

volatile unsigned int* TIMER_STATUS = (unsigned int *) INTERVAL_TIMER_BASE;
volatile unsigned int* TIMER_CTRL = (unsigned int *) (INTERVAL_TIMER_BASE + 4);
volatile unsigned int* PS2_DATA = (unsigned int *) PS2_PORT_BASE;

// Question 1.1
void activer_interruptions() {
	unsigned int tmp;
	NIOS2_READ_STATUS(tmp);
	tmp |= 1;
	NIOS2_WRITE_STATUS(tmp);
}

// Question 1.2
void activer_interruption(int num) {
	unsigned int tmp;
	NIOS2_READ_IENABLE(tmp);
	tmp |= (1 << num);
	NIOS2_WRITE_IENABLE(tmp);
}

int x_pos = 100, y_pos = 100;
char pnext = 0, p[3];
int middle_pos = 0;

void tick() {
	middle_pos = MOD(middle_pos - 1, 320);
}

// Question 3
static void TIMER_ISR(void *context, alt_u32 id) {
	tick();
	*TIMER_STATUS = 0;
}

// Question 4.2
static void MOUSE_ISR(void *context, alt_u32 id)  {
	p[pnext] = (*PS2_DATA) & 0xFF;
	if(++pnext == 3) {
		x_pos = MOD(x_pos + p[1], 320);
		y_pos = MOD(y_pos - p[2], 240);
		pnext = 0;
	}
}

int main() {
	init();
	
	// Question 1.3
	activer_interruptions();
	
	// Question 2.2
	*TIMER_CTRL = 7;
	
	// Question 2.3
	activer_interruption(INTERVAL_TIMER_IRQ);
	
	// Question 4.1
	activer_interruption(PS2_PORT_IRQ);
	
	// Enregistrement des routines d'interruption.
	alt_irq_register ( INTERVAL_TIMER_IRQ, 0, TIMER_ISR );
	alt_irq_register( PS2_PORT_IRQ, 0, MOUSE_ISR );
	
	while(1) {
		clear_screen();
		draw_image((unsigned short *)nuages_img, 320, 240, middle_pos-320, 0);
		draw_image((unsigned short *)nuages_img, 320, 240, middle_pos, 0);
		draw_piou(x_pos, y_pos);
		swap_buffers();
	}
}
\end{lstlisting}

\end{document}
