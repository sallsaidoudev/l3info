\documentclass[a4paper]{article}

\usepackage{a4wide}
\usepackage[utf8]{inputenc}
\usepackage[T1]{fontenc}
\usepackage[francais]{babel}

\usepackage{amssymb}
\usepackage{graphicx}
\usepackage[usenames,dvipsnames]{color}

\usepackage{hyperref} \urlstyle{sf}
\hypersetup{
  colorlinks=true,
  urlcolor=BlueViolet,
  citecolor=BlueViolet,
  linkcolor=BlueViolet,
}
\DeclareUrlCommand\email{\urlstyle{sf}}

\setlength{\parindent}{0pt}
\usepackage{parskip}

\renewcommand{\thefootnote}{\arabic{footnote}}
\newcounter{nbdefi}
\newenvironment{defi}
  {\stepcounter{nbdefi} \par \textbf{def \thenbdefi}~~~}{}
\newcounter{nblem}
\newenvironment{lem}
  {\stepcounter{nblem} \par \textbf{lem \thenblem}~~~}{}
\newcounter{nbprop}
\newenvironment{prop}
  {\stepcounter{nbprop} \par \textbf{prop \thenbprop}~~~}{}
\newenvironment{ex}{\textit{Exemple}\begin{quote}}{\end{quote}}
\newcommand{\rem}[1]{~\\ \textit{Remarque ---} #1}
\newcommand{\reml}[1]{\textit{Remarque ---} #1}

\usepackage{listings}
\lstset{
  language=C,
  basicstyle=\ttfamily,
  keywordstyle=\color{OliveGreen},
  stringstyle=\color{Bittersweet},
  showstringspaces=false,
  commentstyle=\color{Gray},
  numbers=left,
  numberstyle=\ttfamily\color{Gray},
  frame=l,
  columns=fullflexible,
  rulecolor=\color{Gray},
  tabsize=4,
  extendedchars=true,
  literate=
	{É}{{\'E}}1 {è}{{\`e}}1 {à}{{\`a}}1 {È}{{\`E}}1 {À}{{\`A}}1 {ê}{{\^e}}1 {â}{{\^a}}1 {î}{{\^\i}}1 {ô}{{\^o}}1
	{Ê}{{\^E}}1 {Â}{{\^A}}1 {Î}{{\^I}}1 {Ô}{{\^O}}1 {Û}{{\^U}}1 {ë}{{\"e}}1 {ï}{{\"\i}}1 {ü}{{\"u}}1 {Ë}{{\"E}}1
	{Ï}{{\"I}}1 {Ü}{{\"U}}1 {û}{{\^u}}1 {ç}{{\c c}}1 {Ç}{{\c C}}1 {æ}{{\ae}}1 {Æ}{{\AE}}1 {œ}{{\oe}}1 {Œ}{{\OE}}1
	{é}{{\'e}}1,
}


\lstset{language=Java}

\title{Programmation}
\author{Mickaël Foursov -- L3 Info Rennes 1}
\date{2015 -- 2016, S1}

\begin{document}

\maketitle

\subsection*{TP1}

Sujet : tri de boules Vertes, Blanches et Rouges (tableau de \verb?char? en Java).
\begin{lstlisting}
public class Lecture {
    public static InputStream ouvrir(String path);
    public static boolean finFichier(InputStream f);
    public static void fermer(InputStream f);
    public static char lireChar(InputStream f);
    public static String lireString(InputStream f);
    public static int lireInt(InputStream f) throws NumberFormatException;
    public static int lireIntln(InputStream f) throws NumberFormatException;
    public static double lireDouble(InputStream f) throws NumberFormatException;
    public static double lireDoubleln(InputStream f) throws NumberFormatException;
    public static String lireUnite(InputStream f, boolean lire);
}
public class Ecriture {
    public static OutputStream ouvrir(String path);
    public static void fermer(OutputStream f);
    public static void ecrireChar(OutputStream f, char c);
    public static void ecrireString(OutputStream f, String s);
    public static void ecrireStringln(OutputStream f, String s);
    public static void ecrireStringln(OutputStream f);
    public static void ecrireInt(OutputStream f, int n);
    public static void ecrireInt(OutputStream f, int n, int times);
    public static void ecrireDouble(OutputStream f, double d);
}
\end{lstlisting}
Toutes les méthodes sont surchargées par un équivalent pour l'entrée / sortie standard. L'algo de tri est donné, il faut juste corriger quelques erreurs pour respecter les invariants donnés aussi.

\subsection*{TP2}

Ce TP consiste à trouver un entier inférieur à 100 tel que son carré et son cube contiennent une seule fois chaque chiffre. Pas d'API fournie, faut juste faire un petit algo. Je crois qu'il y avait une autre partie mais elle était chiante, osef.

\subsection*{TP3}

Premier exo : suite des Fourmis (1, 11, 21, 1211, 111221, ...), un peu de parsing et de comptage. Deuxième exo : classement d'entiers entrés en ligne de commande, par insertion. Troisième exo : classement de paires d'entiers, un bon gros copier-coller.

\subsection*{TP4}

\begin{lstlisting}
public class List<T extends SuperT> {
  public List();
  public ListIterator iterator();
  public boolean isEmpty();
  public void clear();
  public void setFlag(T);
  public void addHead(T);
  public void addTail(T);
  public List<T> clone();
  public String toString();
}
public interface Iterator<T> {
  public abstract void goForward();
  public abstract void goBackward();
  public abstract void restart();
  public abstract boolean isOnFlag();
  public abstract void remove();
  public abstract T getValue();
  public abstract T nextValue();
  public abstract void addLeft(T);
  public abstract void addRight(T);
  public abstract void setValue(T);
  public abstract String toString();
}
\end{lstlisting}

\subsection*{TP6}

\begin{lstlisting}
public class BinaryTree<T> {
  public BinaryTree();
  public TreeIterator iterator();
  public boolean isEmpty();
  public int height();
  public String toString();
}
public interface Iterator<T> {
  public abstract void goLeft();
  public abstract void goRight();
  public abstract void goUp();
  public abstract void goRoot();
  public abstract boolean isEmpty();
  public abstract NodeType nodeType();
  public abstract void remove();
  public abstract void clear();
  public abstract T getValue();
  public abstract void addValue(T);
  public abstract void setValue(T);
  public abstract void switchValue(int); // wtf is this I don't know
}
\end{lstlisting}

\end{document}
