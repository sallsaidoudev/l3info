\documentclass[a4paper]{article}

\usepackage{a4wide}
\usepackage[utf8]{inputenc}
\usepackage[T1]{fontenc}
\usepackage[francais]{babel}

\usepackage{amssymb}
\usepackage{graphicx}
\usepackage[usenames,dvipsnames]{color}

\usepackage{hyperref} \urlstyle{sf}
\hypersetup{
  colorlinks=true,
  urlcolor=BlueViolet,
  citecolor=BlueViolet,
  linkcolor=BlueViolet,
}
\DeclareUrlCommand\email{\urlstyle{sf}}

\setlength{\parindent}{0pt}
\usepackage{parskip}

\renewcommand{\thefootnote}{\arabic{footnote}}
\newcounter{nbdefi}
\newenvironment{defi}
  {\stepcounter{nbdefi} \par \textbf{def \thenbdefi}~~~}{}
\newcounter{nblem}
\newenvironment{lem}
  {\stepcounter{nblem} \par \textbf{lem \thenblem}~~~}{}
\newcounter{nbprop}
\newenvironment{prop}
  {\stepcounter{nbprop} \par \textbf{prop \thenbprop}~~~}{}
\newenvironment{ex}{\textit{Exemple}\begin{quote}}{\end{quote}}
\newcommand{\rem}[1]{~\\ \textit{Remarque ---} #1}
\newcommand{\reml}[1]{\textit{Remarque ---} #1}

\usepackage{listings}
\lstset{
  language=C,
  basicstyle=\ttfamily,
  keywordstyle=\color{OliveGreen},
  stringstyle=\color{Bittersweet},
  showstringspaces=false,
  commentstyle=\color{Gray},
  numbers=left,
  numberstyle=\ttfamily\color{Gray},
  frame=l,
  columns=fullflexible,
  rulecolor=\color{Gray},
  tabsize=4,
  extendedchars=true,
  literate=
	{É}{{\'E}}1 {è}{{\`e}}1 {à}{{\`a}}1 {È}{{\`E}}1 {À}{{\`A}}1 {ê}{{\^e}}1 {â}{{\^a}}1 {î}{{\^\i}}1 {ô}{{\^o}}1
	{Ê}{{\^E}}1 {Â}{{\^A}}1 {Î}{{\^I}}1 {Ô}{{\^O}}1 {Û}{{\^U}}1 {ë}{{\"e}}1 {ï}{{\"\i}}1 {ü}{{\"u}}1 {Ë}{{\"E}}1
	{Ï}{{\"I}}1 {Ü}{{\"U}}1 {û}{{\^u}}1 {ç}{{\c c}}1 {Ç}{{\c C}}1 {æ}{{\ae}}1 {Æ}{{\AE}}1 {œ}{{\oe}}1 {Œ}{{\OE}}1
	{é}{{\'e}}1,
}


\lstset{language=C}

\title{Système}
\author{Aomar Maddi -- L3 Info Rennes 1}
\date{2015 -- 2016, S1}

\begin{document}

\maketitle

\section{Shell UNIX}

\subsection{Commandes}

\begin{itemize}
	\item \verb?cd <dir>? se déplacer
	\item \verb?ls <nom>? afficher
	\item \verb?more <nom>?
	\item \verb?less <nom>? idem avec scrolling
	\item \verb?ln (-s) <nom1> <nom2>? créer un lien (symbolique)
	\item \verb?mkdir <nom>? créer un répertoire
	\item \verb?rm -rf <nom>? supprimer
	\item \verb?cp -r <nom1> <nom2>? copier nom1 vers nom2
	\item \verb?cat <nom1> ... <nomn>? concaténer et afficher les fichiers
	\item \verb?echo <message>?
	\item \verb?chmod (ugo )(+-=)(rwx) <nom>? changer les droits d'un fichier
	\item \verb?basename <nom>? nom sans chemin
	\item \verb?dirname <nom>? chemin sans le nom
	\item \verb?pwd? afficher le répertoire courant
	\item \verb?exit? quitter le programme
	\item \verb?read <var1> ... <varn>? lire des variables (0 si OK) 
	\item Redirections : \verb?>? sortie standard, \verb?>>? concatène, \verb?>? entrée standard, \verb?|? redirection ES-SS
	\item Délimiteurs : \verb?'? pas d'interprétation, \verb?"? idem sauf \verb?$? et \verb?`?, \verb?`? interprète le contenu
	\item Caractères spéciaux : \verb?*? joker, \verb|?| un caractère, \verb?\? échappement
\end{itemize}

\subsection{grep et find}

\verb?grep (-r) <regex> <nom>? sort toutes les lignes de \verb?nom? (fichier ou répertoire) contenant l'expression régulière donnée (où \verb?.? joker, \verb?*? étoile de Kleene, \verb?[x-y]? classe de caractères).

\verb?find <dir> (-name <nom> -usr <user> -size <n(okM)> -type (dfl) ...)? affiche les fichers contenus dans le répertoire spécifié en filtrant selon les critères donnés ;\\ \verb?-not -a -o \( \)? permettent de combiner les filtres, \verb?-exec cmd {} \;? exécute la commande donnée sur chaque fichier trouvé.

\subsection{Programmation}

\begin{itemize}
	\item \verb?chmod +x script? (le script commence par \verb?#!/bin/sh?) puis \verb?./script arg1 ... argn?
	\item Identificateurs : \verb?$#? nombre de paramètres, \verb?$*? ou \verb?$@? tous, \verb?$0..n? nom du programme puis paramètres, \verb?$HOME?, \verb?$PATH?, ...
	\item Commentaires : \verb?# commentaire?
	\item Variables : \verb?var=(valeur,var2,`commande`)?, peuvent être initialisées par \verb?read?
	\item Tests : \verb?! -a -o? logique, \verb?-d -f <nom>? existence, \verb?-s <nom>? non vide, \verb?-r -w -x <nom>? droits, \verb?-z -n <chaine>? longueur nulle/non nulle, \verb?<ch1> (!)= <ch2>?,\\ \verb;<n1> -eq -neq -gt -lt -ge -le <n2>; arithmétique
	\item Conditions : \begin{verbatim}
if [ commandes ] # si dernière commande renvoie 0
then             # alors
    commandes
else             # sinon
    commandes
fi
	\end{verbatim}
	\item Itérations : \begin{verbatim}
while [ commandes ] # tant que der = 0
do
    commandes
done
for var
in <liste> # explicite $foo $bar $so ou liste telle que $*
do
    commandes
done
	\end{verbatim}
\end{itemize}


\section{Programmation C}

\subsection{Mémo syntaxique}

\begin{lstlisting}
/* Préprocesseur */
#include <stdio.h> // chargement d'une lib c
#include "libperso.h"
#define CONSTANTE "valeur" // constante de préprocesseur

/* Variables et pointeurs */
int i, j = 0; // types : (unsigned ou signed) char short int long float double
char machin = (char) truc; // cast de types
int* pi = &i; // pi est un pointeur d'int sur i
// le type pointeur neutre est void*, le pointeur nul est NULL
*pi += 1; // déréférencement

/* Tableaux */
float stats[64]; // taille statique
int p[] = {2, 3, 5, 7, 11, 13}; // initialisation
p == &p[0]; // tableaux et pointeurs c'est pareil
p[i] == *(p+i); // et oui
// mais on ne peut pas copier ou affecter n'importe comment (strcpy...)

/* Fonctions */
type fonction(type par1, type par2, ...) {
    instructions;
    return resultat; // de type correct, rien si void
}
// localité de bon sens
// paramètres passés par valeur (pour référence on utilise les pointeurs)
char* exemple(int n, char* str); // prototype
// permet d'utiliser une fonction non déclarée
// à rassembler dans un fichier .h

/* Point d'entrée */
int main(int argc, char* argv[]) { // nb d'arguments commande, leur liste
    ...
    return 0; // valeur de retour si pas d'erreurs
}

/* Chaînes et E/S */
#include <string.h> // une chaîne est un char[] finissant par \0
int strln(char* s); // |s|
char* strcpy(char *s1, char *s2); // s1 <- s2
int strcmp(char *s1, char *s2); // 0 si s1 == s2
#include <stdio.h> // fonctions d'entrée / sortie
int getchar(); // lit le prochain caractère sur stdin
void putchar(char c); // écrit un caractère sur stdout
int printf(char* format, e1, ..., en); // écrit selon le formatage
int scanf(char* format, a1, ..., an); // lit les variables selon le formatage
// format : %d %c %s %lf ...

/* Types énumérés et structurés */
enum couleur {coeur, pique, carreau, trefle};
enum couleur macouleur = pique;
struct carte {couleur c, int val}; // référence récursive possible
struct carte macarte = {pique, 9};
typedef struct {couleur c, int val} carte; // type alias (marche aussi avec enum)
macarte.val = 1;
pcarte->c = trefle; // pour un pointeur vers une structure
size_t sizeof(type t); // donne la taille en octets d'un type

/* Gestion de la mémoire */
void* malloc(size_t s); // retourne un pointeur neutre (cast) vers un espace mémoire
void free(void* p); // libère la mémoire

/* Gestion de fichiers */
#include <stdio.h>
FILE* fopen(char* name, char* mode); // mode r, w, a (append)
int fclose(FILE* f); // 0 si fermé, EOF sinon
int putc(int c, FILE* file); // EOF en cas d'erreur
int getc(FILE* file);
// écrit (lit) nb blocs de taille size de buf vers file
int fwrite(void* buf, int size, int nb, FILE* file);
int fread(void* buf, int size, int nb, FILE* file);
// idem printf et scanf
int fprintf(FILE* file, char* format, e1, ..., en);
int fscanf(FILE* file, char* format, a1, ..., an);
\end{lstlisting}

\subsection{Makefile}

Si un fichier a été modifié plus récemment que la cible, les commandes s'exécutent :
\begin{verbatim}
target: file1 file2 ...
    commands
\end{verbatim}

On rajoute des cibles génériques (\verb?clean?, \verb?all?...) par confort d'utilisation ; on peut aussi définir des variables Shell et tout ce qu'il faut pour avoir un Makefile universel.

\end{document}
