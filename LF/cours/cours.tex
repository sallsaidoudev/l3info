\documentclass[a4paper]{article}

\title{Langages Formels}
\author{Anne Grazon -- L3 Info Rennes 1}
\date{2015 -- 2016, S1}

\usepackage{a4wide}
\usepackage[utf8]{inputenc}
\usepackage[T1]{fontenc}
\usepackage[francais]{babel}

\usepackage{amssymb}
\usepackage{graphicx}
\usepackage[usenames,dvipsnames]{color}

\usepackage{hyperref} \urlstyle{sf}
\hypersetup{
  colorlinks=true,
  urlcolor=BlueViolet,
  citecolor=BlueViolet,
  linkcolor=BlueViolet,
}
\DeclareUrlCommand\email{\urlstyle{sf}}

\setlength{\parindent}{0pt}
\usepackage{parskip}

\renewcommand{\thefootnote}{\arabic{footnote}}
\newcounter{nbdefi}
\newenvironment{defi}
  {\stepcounter{nbdefi} \par \textbf{def \thenbdefi}~~~}{}
\newcounter{nblem}
\newenvironment{lem}
  {\stepcounter{nblem} \par \textbf{lem \thenblem}~~~}{}
\newcounter{nbprop}
\newenvironment{prop}
  {\stepcounter{nbprop} \par \textbf{prop \thenbprop}~~~}{}
\newenvironment{ex}{\textit{Exemple}\begin{quote}}{\end{quote}}
\newcommand{\rem}[1]{~\\ \textit{Remarque ---} #1}
\newcommand{\reml}[1]{\textit{Remarque ---} #1}

\usepackage{listings}
\lstset{
  language=C,
  basicstyle=\ttfamily,
  keywordstyle=\color{OliveGreen},
  stringstyle=\color{Bittersweet},
  showstringspaces=false,
  commentstyle=\color{Gray},
  numbers=left,
  numberstyle=\ttfamily\color{Gray},
  frame=l,
  columns=fullflexible,
  rulecolor=\color{Gray},
  tabsize=4,
  extendedchars=true,
  literate=
	{É}{{\'E}}1 {è}{{\`e}}1 {à}{{\`a}}1 {È}{{\`E}}1 {À}{{\`A}}1 {ê}{{\^e}}1 {â}{{\^a}}1 {î}{{\^\i}}1 {ô}{{\^o}}1
	{Ê}{{\^E}}1 {Â}{{\^A}}1 {Î}{{\^I}}1 {Ô}{{\^O}}1 {Û}{{\^U}}1 {ë}{{\"e}}1 {ï}{{\"\i}}1 {ü}{{\"u}}1 {Ë}{{\"E}}1
	{Ï}{{\"I}}1 {Ü}{{\"U}}1 {û}{{\^u}}1 {ç}{{\c c}}1 {Ç}{{\c C}}1 {æ}{{\ae}}1 {Æ}{{\AE}}1 {œ}{{\oe}}1 {Œ}{{\OE}}1
	{é}{{\'e}}1,
}


\begin{document}

\maketitle

% 10/09/15
\section{Monoïdes libres, langages}

\subsection{Monoïdes}

La théorie des langages est née dans les années 60 de la volonté des linguistes de formaliser la notion de grammaire (des langages naturels). Parmi eux, Noam Chomsky a défini quatre types de grammaires associées à quatre types de langages (type 0, 1, 2 et 3). Dans ce cours, nous étudierons les langages algébriques (type 2) et rationnels (type 3).

\begin{defi}Quelques définitions :
\begin{enumerate}
	\item Alphabet --- Ensemble $\Sigma$ fini non vide de symboles, appelés lettres.
	\item Mot sur $\Sigma$ --- Suite finie de lettres. On définit sa longueur $|u| = n$.
	\item Le mot vide est noté $\epsilon$ ou $1_\Sigma$, $|1_\Sigma| = 0$.
	\item $\Sigma^*$ désigne l'ensemble de tous les mots sur $\Sigma$.
	\item La loi de composition interne sur $\Sigma^*$ notée $\cdot$ est la concaténation, qui est associative et admet $1_\Sigma$ comme élément neutre.
	\item Et ça, c'est un monoïde.
\end{enumerate}
\end{defi}

\begin{defi}
Un monoïde est dit libre lorsque la décomposition d'un élément quelconque en ``éléments de base" suivant sa loi de composition, est unique. $\Sigma^*$ est alors le monoïde libre engendré par $\Sigma$.
\end{defi}
\rem{On voit immédiatement que deux mots sont égaux si et seulement si ils sont de même longueur, et ont leur lettres égales deux à deux. Cette propriété caractérise les monoïdes libres.}

\begin{defi}
$v$ est un facteur de $u \Leftrightarrow \exists \alpha, \beta \in \Sigma^*, u=\alpha v\beta$ ; c'est un facteur droit (resp. gauche) si $\beta=1_\Sigma$ (resp. si $\alpha=1_\Sigma$). C'est un facteur propre si $v\neq u$ et $v\neq 1_\Sigma$.
\end{defi}

\begin{defi}
Pour $x\in\Sigma$, $|\cdot|_x$ est le nombre d'occurrences de $x$ dans un mot ; on définit de même le nombre d'occurrences d'un mot dans un autre.
\end{defi}

\subsection{Langages}

Un langage est un ensemble quelconque de mots ($L\subseteq\Sigma^*$, $L\in\mathcal{P}(\Sigma^*)$). L'union, l'intersection et le complémentaire sont définis intuitivement sur les langages.

\begin{defi}
Les autres opérations usuelles sont le produit $L\cdot M = \{uv|u\in L\textnormal{ et }v\in M\}$, l'étoile de Kleene $L^*=\cup_{n\geqslant 0}L^n$ et l'étoile propre $L^+=L^*\setminus L^0$.
\end{defi}


\section{Grammaires algébriques}

\begin{defi}
Une grammaire est un quadruplet $G=(\Sigma,V,S,P)$ où $\Sigma$ est l'alphabet terminal, $V$ disjoint de $\Sigma$ l'alphabet des non-terminaux, $S\in V$ l'axiome de $G$ et $P\subsetneq V\times (X\cup V)^*$ l'ensemble des règles de production.
\end{defi}

\begin{ex}$G_1 = (\Sigma, V, S, P)$ avec $\Sigma=\{a,b\}$, $V=\{S,T\}$ et les règles de production $P\rightarrow aSb+aT$, $T\rightarrow b$.\end{ex}

\begin{defi}
Une grammaire est dite linéaire (resp. à droite et à gauche) si $P\subset V\times(\Sigma^*\times V\times\Sigma^*\cup\Sigma^*)$ (resp. $\Sigma^*\times V$ and so on).
\end{defi}

\begin{ex}$G_1$ est linéaire.\end{ex}

La dérivation consiste à engendrer un mot à partir d'un autre en suivant une règle de production. Elle est notée $\rightarrow$, sa fermeture réflexive $\rightarrow^*$ et une dérivation à l'ordre $n$, $\rightarrow^n$.

\begin{defi}
Le langage engendré par une grammaire est $L(G) = \{f\in\Sigma^* | S\rightarrow^*f\}$ (le langage élargi accepte aussi $V^*$). Réciproquement, un langage est dit algébrique s'il existe une grammaire $G$ telle que $L=L(G)$.
\end{defi}
\rem{Pour prouver qu'une grammaire engendre un langage, on doit donc vérifier deux inclusions.}

\begin{ex}$L(G_1)=\{a^nb^n | n\geqslant1\}$\end{ex}

% 14/09
La famille des langages algébriques sur un alphabet $\Sigma$ est notée $\textsl{Alg}(\Sigma^*)$.

\begin{lem}
Lemme fondamental : soit $G$ une grammaire, $f\in(\Sigma\cup V)^*$. Si $f$ se factorise en $f_0S_1f_1...S_kf_k$ où $f_i\in\Sigma^*$ et $S_i\in V$, alors pour tout $g\in(\Sigma\cup V)^*$, $$f\rightarrow^*g \Leftrightarrow g=f_0h_1f_1...h_kf_k \textnormal{ et }\forall i S_i\rightarrow^*h_i$$
Plus précisément, $f\rightarrow^ng$ si idem et $\forall i S_i\rightarrow^{n_i}h_i$ avec $\sum n_i = n$.
\end{lem}

\begin{prop}
Principe de récurrence : soit $S\subset\mathbb{N}$ telle que $0\in S$ et $\forall n, n\in S \Rightarrow n+1\in S$. Alors $S=\mathbb{N}$.
\end{prop}
\rem{En appliquant ce principe à une propriété $\mathcal{P}(n)$ pour $n$ entier, on peut démontrer des trucs. Il existe aussi la version dite forte de la récurrence.}

\begin{defi}
$A$ est un arbre de dérivation pour une grammaire $G$ si les étiquettes de $A$ sont dans $\Sigma\cup V\cup\{\epsilon\}$, les $\epsilon$ n'ont pas de frères et les n\oe uds $E$ de fils $e_1,...,e_n$ sont tels que $E\rightarrow e_1,...,e_n$ est une règle de production de $G$.
\end{defi}
\rem{Les n\oe uds internes sont donc nécessairement étiquetés dans $V$.}

\begin{defi}
Une grammaire est ambiguë si elle génère des mots qui possèdent plusieurs arbres de dérivation distincts.
\end{defi}

On peut choisir de dériver à gauche ou à droite pour gérer l'ambiguïté.

% 21/09
\begin{prop}
Si $L$ et $M$ sont algébriques, alors $L\cup M$ est algébrique. \'Etant données deux grammaires engendrant $L$ et $M$, on en construit une qui engendre $L\cup M$ par union des règles de production.
\end{prop}

Clôture des algébriques : $\textnormal{Alg}(\Sigma^*)$ est clos par union, produit et étoile mais pas par intersection ni complémentaire.


\section{Automates finis}

\begin{defi}
Un automate fini est un quintuplet $A=(\Sigma,Q,q_0,F,\delta)$ où $\Sigma$ est l'alphabet d'entrée, $Q$ l'ensemble des état, $q_0\in Q$ l'état initial, $F\subseteq Q$ l'ensemble des états finals et $\delta\subseteq Q\times\Sigma\times Q$ l'ensemble des transitions.
\end{defi}

\begin{defi} Quelques définitions :\begin{itemize}
  \item Un chemin est une suite de transitions cohérentes ; sa trace est la suite de ses étiquettes $(x_1,...,x_n)\in \Sigma^n$.
  \item Un mot est reconnu par un automate s'il est la trace d'un chemin menant de l'état initial vers un état final.
  \item Un langage est reconnaissable s'il existe un automate fini qui reconnaît tous ses mots.
\end{itemize}\end{defi}
\reml{Les langages vide et $\Sigma^*$ sont trivialement reconnaissables.}

\begin{prop}
La famille $\textnormal{Rec}(\Sigma^*)$ contient les parties finies de $\Sigma^*$ et est close par union (trivial suivant les définitions), produit et étoile.
\end{prop}

% 28/09
\begin{prop}
Toute grammaire linéaire droite engendre un langage reconnaissable par un automate fini, et réciproquement.
\end{prop}

\begin{defi}
Un automate $A$ est dit déterministe si $\forall q\in Q, \forall x\in \Sigma, \#\{q'\in Q | (q, x, q') \in \delta\} \leqslant 1$ et complet si $\geqslant 1$.
\end{defi}
\rem{Pour un automate déterministe complet, $\delta$ est une fonction totale de $Q\times\Sigma\rightarrow Q$ qui s'étend récursivement et intuitivement en $\hat{\delta}:Q\times\Sigma^*\rightarrow Q$.}
\rem{Complétion d'automate : tout automate fini est équivalent à un automate complet qui peut se construire en ajoutant un état "puits" vers lequel on redirige toutes les transitions manquantes.}

Clôture des reconnaissables : $\textnormal{Rec}(\Sigma^*)$ est clos par étoile (pour un automate donné, on renvoie vers l'état initial les transitions pointant vers un état final) et par intersection (pour deux automates, on construit l'automate des couples d'états, la fonction de transition étant inférée).

\subsection{Lemme de l'étoile}
\begin{lem}
Soit $L$ un langage reconnaissable ; $\exists N$ tel que, $\forall w\in L, |w|\geqslant N$, il existe une factorisation de $w=xyz$ telle que $|x|>0$ et $\forall n, xy^nz\in L$.
\end{lem}
\rem{La preuve se construit à partir d'un automate, en posant $N = \#Q$.}

\begin{prop}
Le langage $\{a^nb^n|n\geqslant 0\}$ n'est pas reconnaissable.
\end{prop}

\subsection{Théorème de Kleene}
\begin{defi}
Rat$(\Sigma^*)$ est la plus petite famille de langages contenant les parties finies de $\Sigma^*$, close par union, intersection et étoile.
\end{defi}

\begin{prop}
Tout langage rationnel ($\in$Rat$(\Sigma^*)$) possède une expression sous la forme d'un nombre fini d'unions, de produits et d'étoiles de langages finis. (On définit par suite les expressions rationnelles.)
\end{prop}

\begin{prop}
Théorème de Kleene --- Rat$(\Sigma^*) = \textnormal{Rec}(\Sigma^*)$
\end{prop}
\rem{Le sens réciproque s'obtient par exemple, par résolution de systèmes avec le lemme d'Arden ($X=AX\cup B \Rightarrow X=A^*B$).}

\subsection{Résiduels}

Diapo 6, pp. 11-24.

\begin{prop}
L'automate des résiduels de $L$ est minimal pour $L$.
\end{prop}

\subsection{Nérode}

\begin{defi}
Un mot de $\Sigma^*$ sépare deux états $q$ et $q'$ d'un automate s'il mène à un état final par un, et non final par l'autre. Deux états sont Nérode-équivalents si aucun mot ne les sépare.
\end{defi}

\begin{defi}
L'automate quotient est l'automate des classes de Nérode (en particulier, $\bar{\delta} = \{(\bar{q},x,\bar{q'})|\exists q\in\bar{q}, q'\in\bar{q'} \textnormal{ tels que } (q,x,q')\in\delta\}$). Il est équivalent à l'automate initial et minimal (puisque $\#Res(L)=\#\{L_q|q\in Q\}=\#\{\bar{q}|q\in Q\}$).
\end{defi}

On le construit par induction pour trouver l'équivalence de Nérode : $\sim_0$ sépare les états finals et non-finals, $\sim_i$ se déduit de $\sim_{i-1}$ en séparant ses classes selon les lettres de $\Sigma$.


\section{Automates à pile}

\begin{defi}
Un automate à pile simple est un quadruplet $(\Sigma, \Gamma, \gamma\in\Gamma, \lambda :\Sigma\times\Gamma\rightarrow\mathcal{P}(\Gamma^*))$. $\gamma$ est le symbole de fond de pile.
\end{defi}

\begin{defi}
Un mot est dit reconnu par un automate à pile simple, si la pile est vide après lecture de ce mot.
\end{defi}

\begin{defi}
Une grammaire algébrique est sous forme normale de Greibach si $P\subset V\times XV^*$.
\end{defi}

\begin{prop}
Tout langage algébrique propre (ne contenant pas le mot vide) possède une grammaire sous forme normale de Greibach.
\end{prop}

\begin{prop}
Un langage propre est algébrique si, et seulement si il existe un automate à pile simple qui le reconnaît.
\end{prop}
\rem{On construit l'automate directement à partir de la grammaire, et réciproquement.}

\subsection{Automates à pile généraux}

\begin{defi}
Même chose avec un ensemble fini d'états, dont un initial et éventuellement des finals.
\end{defi}

\begin{defi}
Ces automates peuvent reconnaître un langage par pile vide ou par état final (i.e. un calcul valide mène à...).
\end{defi}

\begin{prop}
Un langage est algébrique si, et seulement si il existe un automate à pile qui le reconnaît par pile vide (ou par état final, c'est équivalent).
\end{prop}
\rem{Les automates à pile simple ne sont qu'un cas particulier (un seul état et pas d'$\epsilon$-transition).}

\begin{defi}
Un automate à pile est déterministe si tout calcul est entièrement déterminé par l'état, le sommet de pile et la lettre lue.
\end{defi}
\rem{Un langage est déterministe quand un automate déterministe le reconnaît par état final.}

\begin{prop}
Inclusions : Rec$(\Sigma^*)\subset$Det$(\Sigma^*)\subset$Non ambigus$\subset$Alg$(\Sigma^*)$.
\end{prop}

\begin{prop}
L'ensemble des langages algébriques est clos par intersection rationnelle.
\end{prop}

\begin{prop}
Théorème de Bar-Hillel, Perles et Shamir : le lemme de l'étoile en mieux. Pour $L$ algébrique, il existe $N$ tel que tout mot $f\in L, |f|>N$ se décompose en $\alpha u\beta v\gamma$ avec $uv\neq \epsilon$, $|u\beta v|\leqslant N$ et $\forall n, \alpha u^n\beta v^n\gamma\in L$.
\end{prop}

\end{document}
