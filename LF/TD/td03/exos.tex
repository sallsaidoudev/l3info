\documentclass[a4paper]{article}

\usepackage{a4wide}
\usepackage[utf8]{inputenc}
\usepackage[T1]{fontenc}
\usepackage[francais]{babel}

\usepackage{amssymb}
\usepackage{graphicx}
\usepackage[usenames,dvipsnames]{color}

\usepackage{hyperref} \urlstyle{sf}
\hypersetup{
  colorlinks=true,
  urlcolor=BlueViolet,
  citecolor=BlueViolet,
  linkcolor=BlueViolet,
}
\DeclareUrlCommand\email{\urlstyle{sf}}

\setlength{\parindent}{0pt}
\usepackage{parskip}

\renewcommand{\thefootnote}{\arabic{footnote}}
\newcounter{nbdefi}
\newenvironment{defi}
  {\stepcounter{nbdefi} \par \textbf{def \thenbdefi}~~~}{}
\newcounter{nblem}
\newenvironment{lem}
  {\stepcounter{nblem} \par \textbf{lem \thenblem}~~~}{}
\newcounter{nbprop}
\newenvironment{prop}
  {\stepcounter{nbprop} \par \textbf{prop \thenbprop}~~~}{}
\newenvironment{ex}{\textit{Exemple}\begin{quote}}{\end{quote}}
\newcommand{\rem}[1]{~\\ \textit{Remarque ---} #1}
\newcommand{\reml}[1]{\textit{Remarque ---} #1}

\usepackage{listings}
\lstset{
  language=C,
  basicstyle=\ttfamily,
  keywordstyle=\color{OliveGreen},
  stringstyle=\color{Bittersweet},
  showstringspaces=false,
  commentstyle=\color{Gray},
  numbers=left,
  numberstyle=\ttfamily\color{Gray},
  frame=l,
  columns=fullflexible,
  rulecolor=\color{Gray},
  tabsize=4,
  extendedchars=true,
  literate=
	{É}{{\'E}}1 {è}{{\`e}}1 {à}{{\`a}}1 {È}{{\`E}}1 {À}{{\`A}}1 {ê}{{\^e}}1 {â}{{\^a}}1 {î}{{\^\i}}1 {ô}{{\^o}}1
	{Ê}{{\^E}}1 {Â}{{\^A}}1 {Î}{{\^I}}1 {Ô}{{\^O}}1 {Û}{{\^U}}1 {ë}{{\"e}}1 {ï}{{\"\i}}1 {ü}{{\"u}}1 {Ë}{{\"E}}1
	{Ï}{{\"I}}1 {Ü}{{\"U}}1 {û}{{\^u}}1 {ç}{{\c c}}1 {Ç}{{\c C}}1 {æ}{{\ae}}1 {Æ}{{\AE}}1 {œ}{{\oe}}1 {Œ}{{\OE}}1
	{é}{{\'e}}1,
}

\newcommand{\ct}{\textnormal}
\newcommand{\la}{\ct{a}}
\newcommand{\lb}{\ct{b}}
\newcommand{\blank}{\hspace*{3em}}

\begin{document}

\section*{LF -- TD 3}

\paragraph{Exercice 3}~\\
Montrer que $L=\{uv | u,v\in\Sigma^* \wedge |u| = |v| \wedge u\neq v\}$ est engendré par la grammaire $G$ suivante :\\
$S\rightarrow AB|BA\\
A\rightarrow aAa|aAb|bAa|bAb|a\\
B\rightarrow aBa|aBb|bBa|bBb|b$

Montrons que $L(G) \subset L$ :

Soit $w\in L(G)$. D'après les règles de production, $\exists f,f',g,g' \in\Sigma^*$ tels que $w=faf'gbg'$ (ou $w=fbf'gag'$, et la démonstration est symétrique) avec $|f|=|f'|=n$ et $|g|=|g'|=p$.

$w$ étant alors de longueur paire, soient $u$ et $v$ de même longueur tels que $w=uv=faf'gbg'$ ; on a $|w|=2n+2p+2$ et donc $|u|=|v|=n+p+1$. Ainsi, la lettre de $u$ en position $(n+1)$ est un $a$, alors que celle de $v$ à la même position est un $b$ : on a bien $u\neq v$ et donc $w\in L$.

Réciproquement, pour $L \subset L(G)$, on dérive un mot de $L$ en s'appuyant sur la première lettre différente entre ses moitiés.

\paragraph{Exercice 4}~\\
1. $\epsilon, ab, aabb \in L_1=\{a^nb^n|n\geqslant 0\}$ ; $\epsilon, abab, abaabb \in L_2=L_1\cdot L_1$ ;\\
$aabbababab, aaaaabbbbb, aaabbbaabb \in L_3=L_1^*$.

2. $G_1 : S\rightarrow aSb|\epsilon$ ; $G_2 : S\rightarrow TT, T\rightarrow aTb|\epsilon$ ; $G_3 : S\rightarrow SS | T, T\rightarrow aSb|\epsilon$.

3. et 4. La question précédente fournit les constructions de grammaires pour le produit et l'étoile de langages.
\end{document}
