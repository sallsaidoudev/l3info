\documentclass[a4paper]{article}

\usepackage{a4wide}
\usepackage[utf8]{inputenc}
\usepackage[T1]{fontenc}
\usepackage[francais]{babel}

\usepackage{amssymb}
\usepackage{graphicx}
\usepackage[usenames,dvipsnames]{color}

\usepackage{hyperref} \urlstyle{sf}
\hypersetup{
  colorlinks=true,
  urlcolor=BlueViolet,
  citecolor=BlueViolet,
  linkcolor=BlueViolet,
}
\DeclareUrlCommand\email{\urlstyle{sf}}

\setlength{\parindent}{0pt}
\usepackage{parskip}

\renewcommand{\thefootnote}{\arabic{footnote}}
\newcounter{nbdefi}
\newenvironment{defi}
  {\stepcounter{nbdefi} \par \textbf{def \thenbdefi}~~~}{}
\newcounter{nblem}
\newenvironment{lem}
  {\stepcounter{nblem} \par \textbf{lem \thenblem}~~~}{}
\newcounter{nbprop}
\newenvironment{prop}
  {\stepcounter{nbprop} \par \textbf{prop \thenbprop}~~~}{}
\newenvironment{ex}{\textit{Exemple}\begin{quote}}{\end{quote}}
\newcommand{\rem}[1]{~\\ \textit{Remarque ---} #1}
\newcommand{\reml}[1]{\textit{Remarque ---} #1}

\usepackage{listings}
\lstset{
  language=C,
  basicstyle=\ttfamily,
  keywordstyle=\color{OliveGreen},
  stringstyle=\color{Bittersweet},
  showstringspaces=false,
  commentstyle=\color{Gray},
  numbers=left,
  numberstyle=\ttfamily\color{Gray},
  frame=l,
  columns=fullflexible,
  rulecolor=\color{Gray},
  tabsize=4,
  extendedchars=true,
  literate=
	{É}{{\'E}}1 {è}{{\`e}}1 {à}{{\`a}}1 {È}{{\`E}}1 {À}{{\`A}}1 {ê}{{\^e}}1 {â}{{\^a}}1 {î}{{\^\i}}1 {ô}{{\^o}}1
	{Ê}{{\^E}}1 {Â}{{\^A}}1 {Î}{{\^I}}1 {Ô}{{\^O}}1 {Û}{{\^U}}1 {ë}{{\"e}}1 {ï}{{\"\i}}1 {ü}{{\"u}}1 {Ë}{{\"E}}1
	{Ï}{{\"I}}1 {Ü}{{\"U}}1 {û}{{\^u}}1 {ç}{{\c c}}1 {Ç}{{\c C}}1 {æ}{{\ae}}1 {Æ}{{\AE}}1 {œ}{{\oe}}1 {Œ}{{\OE}}1
	{é}{{\'e}}1,
}


\newcounter{noexo}
\newcommand{\exo}{\stepcounter{noexo} \paragraph{Exercice \thenoexo} ~\\ ~\\}

\title{Méthodes Algorithmiques -- TD}
\author{Sophie Pinchinat -- L3 Info Rennes 1}
\date{2015 -- 2016, S2}

\begin{document}

\maketitle

\subsection*{TD 0}

\exo
\begin{tabular}{c|ccccccccc}
	Q & 1 & 2 & 3 & 4 & 5 & 6 & 7 & 8 & 9 \\
	R & $n$ & $n+1$ & $n$ & $2n$ & $n^2$ & $n^3$ & $n^3+n^2+2n$ & $\frac{n(n+1)}{2}$ & $+\infty$
\end{tabular}

\exo
1. Soit $f$ telle que $f\in O(g)$ ; alors $\exists c_1>0$ tq $f(n) \leqslant c_1g(n)$ pour $n>n_0$ et de plus comme $g \in \Theta(h)$, $\exists c_2, c_3>0$ tq $c_2g(n) \leqslant h(n) \leqslant c_3g(n)$ pour $n>n_1$. Il vient alors pour tout $n>\max(n_0, n_1), f(n)\leqslant c_1g(n) \leqslant \frac{c_1}{c_2}h(n)$ et donc on a bien $f\in O(h)$.

2. Démonstration analogue.

3. Posons $f(n)=3n^2, g(n)=n^2 et h(n)=e^n$ ; alors on a bien $f\in\Theta(g)$ et $g\in O(h)$ mais pourtant pas $f\in\Theta(h)$ (quelque soit $c$, il existe un $n_0$ à partir duquel $3cn^2 < e^n$, c'est assez convaincant).

\exo
1. Avant une étape de la boucle, $s=\sum_{j=0}^{i-1}A[j]$ ; la boucle donne $s'=\sum_{j=0}^{i-1}A[j] + A[i] = \sum_{j=0}^{i}A[j]$ et $i'=i+1$ d'où on a bien $s'=\sum_{j=0}^{i'-1}A[j]$.

2. Avant la boucle, $s=i=0$ donc l'invariant est trivialement vérifié.

3. Comme de plus chaque étape de boucle maintient l'invariant, et que la boucle se termine avec $i=n$, on a bien à la fin $s=\sum_{j=0}^{n-1}A[j]$. L'algorithme est correct.

4. La boucle principale s'exécute $n$ fois pour un tableau de taille $n$, d'où le résultat immédiat.

\exo
1. Maximum d'un tableau $A$ de taille $n$ :
\begin{quote}\begin{verbatim}
1. m = A[0]; k = 0; i = 1
2. tant que i < n faire
3.     si A[i] > m
4.         m = A[i]
5.         k = i
6. retourner (m, k)
\end{verbatim}\end{quote}

2. $m=\max_{0\leqslant j<n}A[j] \wedge A[k] = m$

3. Soit $m=\max_{0\leqslant j<i}A[j] \wedge A[k] = m$ notre invariant. Avant la boucle, il est vérifié : on a bien $m=\max_{0\leqslant j<1}A[j]=A[0]$ et $A[k]=A[0]=m$. En le supposant vrai au début d'une itération : \begin{itemize}
	\item soit $A[i]\leqslant m$, les valeurs ne changent pas et l'invariant est toujours vrai (la valeur initiale de $m$ est toujours le maximum courant) ;
	\item soit $A[i]>m$ (la valeur lue est supérieure au maximum connu) et alors les valeurs sont mises à jour : $m$ est le nouveau maximum et $k=i\Rightarrow A[k]=m$, et l'invariant est de nouveau vrai après l'itération.
\end{itemize}

4. L'invariant étant vérifié, à la fin de la boucle on a bien $m=\max_{0\leqslant j<n}A[j] \wedge A[k] = m$ comme demandé en question 2. L'algorithme fonctionne.

5. Complexité :\\
\begin{tabular}{c|ccc}
	Op & Comparaison & Affectation & Les deux \\
	$T(n)$ & $n$ & $O(n)$ & $O(n)$
\end{tabular}

\exo
1. On peut diviser successivement le nombre par 2 en notant les restes à l'envers. Ou alors sinon on fait autre chose.

(...)

\exo
1. Calcul de la puissance $n$ de $a$ :
\begin{quote}\begin{verbatim}
1. r = a; i = 1
2. tant que i < n faire
3.     r = r*a
4. retourner r
\end{verbatim}\end{quote}

2. Un invariant : $r=a^i$ (vrai au début avec $r=a^1=a$, dans la boucle avec $r'=ra=a^ia=a^{i+1}=a^{i'}$).

3. Cet algorithme a une complexité de $n$.

4. La boucle itère $\lceil\log_2 m\rceil$ fois.

5. Trivialement vrai avant la boucle, puis si $m$ impair : res$'b'^{m'}=$ res$\times b(b^2)^{\lfloor m/2\rfloor}=$ res$\times bb^{m-1}=a^n$ et si $m$ pair, res$'b'^{m'}=$ res$\times(b^2)^{\lfloor m/2\rfloor}=$ res$\times b^m=a^n$.

6. L'invariant étant vérifié, à la fin de la boucle (pour $m=0$) il l'est toujours d'où res = res$\times b^0 = a^n$ : l'algorithme est correct.

\exo
1. Au départ de la boucle, $res=\prod_{i=n+1}^{n}i=1$ ; au cours d'une itération, $$res'=res\times m=m\prod_{i=m+1}^{n}i=\prod_{i=m}^{n}i=\prod_{i=m'+1}^{n}i$$ (puisque $m'=m-1$).

2. En fin de calcul on a donc $res=\prod_{i=1}^{n}i=n!$ ; le programme est correct.

3. Complexité : $T(n)=n$.

\subsection*{TD 1}
\setcounter{noexo}{0}

\exo
Complexité : $O(\log_2 n)$

\exo
1. Résolution récursive des tours de Hanoï :
\begin{quote}\begin{verbatim}
1. HANOI n ori dest buf :
2. si n = 1
3.     déplacer ori vers dest
4. sinon
5.     HANOI n-1 ori buf dest
6.     déplacer ori vers dest
7.     HANOI n-1 buf dest ori
\end{verbatim}\end{quote}

2. On a la relation $T(n)=2T(n-1)+1, T(1)=1$ ce qui donne pour complexité $O(2^n)$.

\exo
1. Pour $n=2 (r=1)$, on a bien $Pav(T)$ : dans la table à trou de dimension 2, où que soit le trou, il est complémentaire à un des quatre triominos par définition.

Supposons maintenant $Pav(T)$ pour $n=2^r$ ; alors, pour $n'=2^{r+1}$, toute table a un trou dans un de ses quadrants de taille $2^r$ et ce quadrant peut être pavé. On observe en outre que les trois quadrants restants peuvent être considérés comme trois tables à trou disposées en L, avec leurs trous accolés au centre. Ces trois quadrants peuvent être pavés, et le trou au centre a également la forme d'un triomino : on a bien $Pav(T)$ pour $n'=2^{r+1}$.

2. Algorithme de pavage :
\begin{quote}\begin{verbatim}
1. PAV r i j :
2. si r = 1
3.     placer le bon triomino
4. sinon
5.     paver le quadrant troué
6.     paver les trois autres 
7.     boucher le dernier trou
\end{verbatim}\end{quote}

3. Il faut que $n^2\equiv 1[3]$, et alors le nombre de triominos utilisé est $\sqrt{\frac{n^2-1}{3}}$.

\exo
\begin{verbatim}
bool dans-triangle (Point A, Point B, Point C, Point p):
    renvoyer même-côté (A, B, C, p)
        && même-côté (B, C, A, p)
        && même-côté (C, A, B, p)
\end{verbatim}

\end{document}
