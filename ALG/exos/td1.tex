\documentclass[a4paper]{article}

\usepackage{a4wide}
\usepackage[utf8]{inputenc}
\usepackage[T1]{fontenc}
\usepackage[francais]{babel}

\usepackage{amssymb}
\usepackage{graphicx}
\usepackage[usenames,dvipsnames]{color}

\usepackage{hyperref} \urlstyle{sf}
\hypersetup{
  colorlinks=true,
  urlcolor=BlueViolet,
  citecolor=BlueViolet,
  linkcolor=BlueViolet,
}
\DeclareUrlCommand\email{\urlstyle{sf}}

\setlength{\parindent}{0pt}
\usepackage{parskip}

\renewcommand{\thefootnote}{\arabic{footnote}}
\newcounter{nbdefi}
\newenvironment{defi}
  {\stepcounter{nbdefi} \par \textbf{def \thenbdefi}~~~}{}
\newcounter{nblem}
\newenvironment{lem}
  {\stepcounter{nblem} \par \textbf{lem \thenblem}~~~}{}
\newcounter{nbprop}
\newenvironment{prop}
  {\stepcounter{nbprop} \par \textbf{prop \thenbprop}~~~}{}
\newenvironment{ex}{\textit{Exemple}\begin{quote}}{\end{quote}}
\newcommand{\rem}[1]{~\\ \textit{Remarque ---} #1}
\newcommand{\reml}[1]{\textit{Remarque ---} #1}

\usepackage{listings}
\lstset{
  language=C,
  basicstyle=\ttfamily,
  keywordstyle=\color{OliveGreen},
  stringstyle=\color{Bittersweet},
  showstringspaces=false,
  commentstyle=\color{Gray},
  numbers=left,
  numberstyle=\ttfamily\color{Gray},
  frame=l,
  columns=fullflexible,
  rulecolor=\color{Gray},
  tabsize=4,
  extendedchars=true,
  literate=
	{É}{{\'E}}1 {è}{{\`e}}1 {à}{{\`a}}1 {È}{{\`E}}1 {À}{{\`A}}1 {ê}{{\^e}}1 {â}{{\^a}}1 {î}{{\^\i}}1 {ô}{{\^o}}1
	{Ê}{{\^E}}1 {Â}{{\^A}}1 {Î}{{\^I}}1 {Ô}{{\^O}}1 {Û}{{\^U}}1 {ë}{{\"e}}1 {ï}{{\"\i}}1 {ü}{{\"u}}1 {Ë}{{\"E}}1
	{Ï}{{\"I}}1 {Ü}{{\"U}}1 {û}{{\^u}}1 {ç}{{\c c}}1 {Ç}{{\c C}}1 {æ}{{\ae}}1 {Æ}{{\AE}}1 {œ}{{\oe}}1 {Œ}{{\OE}}1
	{é}{{\'e}}1,
}


\newcounter{noexo}
\newcommand{\exo}{\stepcounter{noexo} \paragraph{Exercice \thenoexo}}

\title{Méthodes Algorithmiques -- Exercices 1}
\author{Sophie Pinchinat -- L3 Info Rennes 1}
\date{2015 -- 2016, S2}

\begin{document}

\maketitle

% 10/09/15
\section{TD1}

\exo Donner trois exemples de fonctions $O(n^2)$.\\
$f_1(n)=42n^2, f_2(n)=\Pi, f_3(n)=3n^2+n+4$

\exo Déterminer la classe de complexité de $f(n)=3n+2\frac{e^n}{n^4}$.\\
$f(n) = O(e^n)$

\exo Soient $g$ et $h$ telles que $g(n) = O(h(n))$, montrer que $\forall f, f(n) = O(g(n) \Rightarrow f(n) = O(h(n))$.\\
Soit $f$ telle que $f(n) = O(g(n))$ ; alors $\exists c_1>0$ tel que $f(n)\leqslant c_1g(n)$ à partir d'un certain rang $n_1$ ; de plus comme $g(n)=O(h(n)), \exists c_2>0$ tel que $g(n)\leqslant c_2h(n)$ à partir de $n_2$. Alors $$\forall n > \textnormal{max}(n_1, n_2), f(n)\leqslant c_1g(n) \leqslant c_1c_2h(n)$$ et donc $f(n) = O(h(n))$.

\exo Le lemme \og la fusion prend un temps linéaire \fg\ se traduit formellement comme suit : pour deux tableaux $T_1$ et $T_2$ de tailles respectives $m$ et $n$, $M(T_1, T_2)$ s'exécute en $O(m+n)$.

\exo Un algorithme de type \og Diviser pour Régner \fg\ obéit à la formule de complexité suivante : $T(n) = \Theta(1)$ si $n=1$, $aT(\lceil n/b \rceil) + cn^d$ sinon. Cette formule se déplie récursivement autant de fois qu'on peut diviser $n$ par $b$, soit $log_b n$, en ajoutant un facteur $a$ à chaque fois d'où au final dans le résultat un facteur $a ^{log_b n} = n ^{log_b a}$.

\end{document}
